\documentclass[]{friggeri-cv} % Add 'print' as an option into the square bracket to remove colors from this template for printing
\usepackage{pdfpages}
\usepackage{lipsum}
\usepackage{graphicx}

\begin{document}
\header\Large{Dheeraj Kumar Joshi}{SRE}
%--------------------------------------------------------------------------------------
%	SIDEBAR SECTION
%----------------------------------------------------------------------------------------

\begin{aside} % In the aside, each new line forces a line break
\includegraphics[scale=0.5]{picture.jpg}
\section{Current Address}
323-A, Sengkang
Singapore, 541323.
~
\section{Contact}
Mobile: +65-93554122
dheerajkrjoshi3@gmail.com
~
\section{Social Profile}
\href{https://www.linkedin.com/in/dheerajkrjoshi/}{LinkedIn:dheerajkrjoshi}
\href{https://github.com/DheerajJoshi}{GitHub:DheerajJoshi}
\textbf{}
\section{Programming}
Python, Shell
\textbf{}
\section{Skills}
Linux, Docker, AWS, Azure, GCP,
CI/CD, Ansible,
Cloudformation, Terraform,
Python, Shell scripting, Monitoring, Git, Disaster Recovery, ELK
\textbf{}
\section{Tools and Software}
Slack, JIRA, DataDog, New Relic, JumpCloud, Pagerduty, Atom, Terraform Enterprise, Jenkins, Phabricator, Conveyer 
\textbf{}
\section{Fields}
SRE \& DevOps
\textbf{}
\end{aside}

%----------------------------------------------------------------------------------------
%	Summary
%----------------------------------------------------------------------------------------
\section{Objective and Professional Snapshot}
{Seeking a varied and challenging \textbf{\textit{System Engineer(Site Reliability Engineer)}} position requiring new and creative applications of technology to solve problems at scale.}

{Experienced SRE/Devops committed to maintaining cutting edge technical skills and knowledge towards the Infra as code, Configuration management and monitoring along with the real time events processing to design the automated healing system for better reliability and faster shipping of the micro-services/applications without compromising the overall stability of the infrastructure. Expertise in requirement gathering and analysis, architectural, component and interface design and development for applications. Professional, detail-oriented SRE motivated to drive projects from start to finish as part of a dynamic team.}

%----------------------------------------------------------------------------------------
%	WORK EXPERIENCE SECTION
%----------------------------------------------------------------------------------------

\section{Experience}

\begin{entrylist}
%------------------------------------------------
\entry
{\emph{Site Reliability Engineer - Cluster Management}}
{\href{https://www.grab.com/sg/}{Grab, Singapore}}
{March 2018(SysOps Engineer) - August 2018) September 2018(SRE) - ongoing}
{Works with a team of 20 SRE engineers from various domains experts. Especially working with service teams. Implemented everything as infrastructure as code from scratch for various environments. Managing the 396+ micro services environment over the AWS public cloud for a highly scalable applications having multiple agents and building and deploying of the application in a highly available environment over the cloud infrastructure.
\begin{itemize}
\item {Engage in and improve the whole lifecycle of services(from scratch to live and beyond), from inception and design, through to deployment, operation and refinement.}
\item {Support services before they go live through activities such as system design consulting, developing complete infra for micro-service, developing software platforms and frameworks, capacity planning and launch reviews, troubleshooting all the ways to find issues.}
\item {Maintain services once they are live by measuring and monitoring availability, latency and overall system health.}
\item {Rotational on-call engineer for complete infrastructure(396+ micro-services). Works with each service team(tech-families) for making service reliable. Respond to incidents and write blameless postmortems with respective service teams.}
\item {Works with service teams on cost-management for their service, by understand service and ask them to change resources accordingly based on utilization of resources by service. Helping them to maintain their cost budget for various services.}
\item {Works with service teams to decommission old services which is no more in use effectively by creating automation for them.}
\item {Automated(using python) job to get information who has what access on service at instance level. By running job with service name and details uploaded in S3 bucket in service directory, which is access by audit team.}
\item {Created pre-prod and pre-staging environment as a part of stability testing}
\item {Provide training to each tech-family for the best practice followed by their respective service team as a part of service reliability(embedded-SRE program).}
\end{itemize}}
\end{entrylist}
\begin{entrylist}
%------------------------------------------------
\entry
{\emph{Jr. Infrastructure Engineer(DevOps)}}
{\href{https://punchh.com/}{Punchh, India}}
{July 2017 - February 2018}
{Works with a team of 2 DevOps engineers.
\begin{itemize}
\item {Created the Punchh Infrastructure in AWS using cloudformation and deploy using stacks in multiple regions.}
\item {Created CI/CD pipelines using Jenkins for canary deployment and integrated real time output on Slack.}
\item {Created slack custom slash commands(Dev-env) for integration of AWS Opswork by lambda function and API Gateway. It helps developers and QA team to create testing infra in staging environment from a single command using slack. Automated all process from Slack so restricted access from console and made process faster.}
\item {Providing support for various Punchh applications hosted in Cloud environment and primary setup for new business.}
\item {Launching fully chef configured and build management system to deploy servers with the proper configuration on a per role and environment basis.}
\item {Automate the creation of base AMI with new updates in packages and security patches using Jenkins.}
\item {Monitoring health of the cloud infrastructure by deploying various monitoring tools such as New Relic and Cloudwatch.}
\item {Created DockerFile and worked with service team for making their app backend to support Docker deployment pipelines.}
\item {Troubleshooting and resolving the incidents. Being on-call for infrastructure and monitor things in shift for esclation level 2}
\item {Provided training to new intern in devops.}
\end{itemize}}
\end{entrylist}
\begin{entrylist}
%------------------------------------------------
\entry
{\emph{Python Developer Trainee and Google Cloud Intern}}
{\href{http://www.celebaltech.com/#/}{CelebalTech(now) previously Celebal, India}}
{May 2016 - June 2017}
{ Only completed POC, not in production.
\begin{itemize}
\item {Master Data Management(MDM): A Master Data Management Platform for Healthcare and Life Sciences organizations. The platform is equipped with advanced data mastering capabilities including real time Data Validation, Mastering Rule Application and Data Stewardship mechanism. The solution also provides analytic layer over our mastered flows in form of real time reports and dashboards. The platform is equipped with advanced data mastering capabilities including real time Data Validation, Mastering Rule Application and Data Stewardship mechanism. The solution also provides analytic layer over our mastered flows in form of real time reports and dashboards.}
\item {Worked as a Python/Django developer and worked at algorithm implementation for
mastering process also responsible for code versioning on Gitlab.}
\end{itemize}}
\end{entrylist}

\section{Certification}

\begin{entrylist}
%------------------------------------------------
\aentry
{June, 2015}
{{Red Hat Certified Engineer - v7}}
{By {Red Hat Inc.}}
\aentry
{June, 2015}
{{Red Hat Certified System Administrator - v7}}
{By {Red Hat Inc.}}
\aentry
{June, 2015}
{{Oracle Certified Associate - 11g}}
{By {Oracle Inc.}}
\aentry
{September, 2015}
{{Red Hat Certified OpenStack Cloud Administrator}}
{By {Red Hat Inc.}}
\end{entrylist}

%----------------------------------------------------------------------------------------
%	EDUCATION SECTION
%----------------------------------------------------------------------------------------

\section{Education}

\begin{entrylist}
%------------------------------------------------
\aentry
{2013 - 2017}
{{Bachelor} {\normalfont of Computer Science}}
{\href{https://aryacollege.in/}{ACEIT, Jaipur, India}}
%------------------------------------------------
\end{entrylist}

\end{document}
